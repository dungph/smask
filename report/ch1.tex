\chapter{GIỚI THIỆU TỔNG QUAN}
\section{Tổng quan về an toàn và bảo mật thông tin}
\subsection{Khái niệm mở đầu}
\paragraph{}
Ngày nay với sự phát triển bùng nổ của công nghệ thông tin, hầu hết các thông tin của doanh nghiệp như chiến lược kinh doanh, các thông tin về khách hàng, nhà cung cấp, tài chính, mức lương nhân viên,…đều được lưu trữ trên hệ thống máy tính. Cùng với sự phát triển của doanh nghiệp là những đòi hỏi ngày càng cao của môi trường kinh doanh yêu cầu doanh nghiệp cần phải chia sẻ thông tin của mình cho nhiều đối tượng khác nhau qua Internet hay Intranet. Việc mất mát, rò rỉ thông tin có thể ảnh hưởng nghiêm trọng đến tài chính, danh tiếng của công ty và quan hệ với khách hàng. 

Hệ thống thông tin là một hệ thống bao gồm các yếu tố có quan hệ với nhau cùng làm nhiệm vụ thu thập, xử lý, lưu trữ và phân phối thông tin và dữ liệu và cung cấp một cơ chế phản hồi để đạt được một mục tiêu định trước. Các thành phần của hệ thống bao gồm phần cứng, phần mềm, mạng truyền dữ liệu, dữ liệu và con người trong hệ thống thông tin.

Các phương thức tấn công thông qua mạng ngày càng tinh vi, phức tạp có thể dẫn đến mất mát thông tin, thậm chí có thể làm sụp đổ hoàn toàn hệ thống thông tin của doanh nghiệp. Vì vậy an toàn và bảo mật thông tin là nhiệm vụ rất nặng nề và khó đoán trước được, nhưng tựu trung lại gồm ba hướng chính: 
\begin{itemize}
    \item Bảo đảm an toàn thông tin tại máy chủ 
    \item Bảo đảm an toàn cho phía máy trạm 
    \item Bảo mật thông tin trên đường truyền 
\end{itemize}

\subsection{Các mối đe dọa và thiệt hại}
\paragraph{}
Ba mối đe dọa chủ yếu đối với hệ thống:
\begin{itemize}
    \item Phá hoại: kẻ thù phá hỏng thiết bị phần cứng hoặc phần mềm hoạt động trên hệ.
    \item Sửa đổi: tài sản của hệ thống bị sửa đổi trái phép. Điều này thường làm cho hệ thống không hoạt động đúng chức năng của nó. Ví dụ như thay đổi mật khẩu, quyền người dùng làm họ không thể truy cập vào hệ thống để làm việc.
    \item Can thiệp: Tài sản bị truy cập bởi những người không có thẩm quyền, các truyền thông thực hiện trên hệ thống bị ngăn chặn, sửa đổi.
thống
\end{itemize}
\subsection{Giải pháp điều khiển và kiểm soát}
\subsection{Mục tiêu và nguyên tắc chung}
\subsection{Ba mục tiêu}
\subsection{Hai nguyên tắc}

\paragraph{}
Tấn công mạng là một trong những vấn đề quan trọng về an toàn và bảo mật thông tin. Đó là những nỗ lực của kẻ tấn công để truy cập vào các hệ thống mạng của một tổ chức hoặc cá nhân mà không có sự cho phép. Những tấn công mạng này có thể gây ra những thiệt hại nghiêm trọng cho các tổ chức hoặc cá nhân, bao gồm mất dữ liệu, mất tiền và thiệt hại đến danh tính.

Phần mềm độc hại: Phần mềm độc hại là một loại phần mềm được thiết kế để gây hại cho hệ thống máy tính hoặc để truy cập vào thông tin cá nhân. Các loại phần mềm độc hại bao gồm virus, phần mềm giám điệp, phần mềm mã độc và phần mềm ransonware.

Xâm nhập: Xâm nhập là quá trình xâm nhập vào hệ thống hoặc mạng của một tổ chức hoặc cá nhân mà không có sự cho phép. Các tấn công xâm nhập này có thể gây ra những thiệt hại nghiêm trọng cho các tổ chức hoặc cá nhân, bao gồm mất dữ liệu và thiệt hại đến danh tiếng. 

Lừa đảo trực tuyến: Lừa đảo trực tuyến là một hoạt động gian lận trực tuyến được thực hiện bằng cách sử dụng các kỹ thuật gian lận để lừa đảo người dùng đưa ra thông tin cá nhân hoặc tiền bạc. 

Rò rỉ dữ liệu: Rò rỉ dữ liệu là quá trình tiết lộ thông tin cá nhân hoặc thông tin nhạy cảm của một cá nhân hoặc tổ chức cho người không có quyền truy cập vào thông tin đó. Rò rỉ dữ liệu có thể gây ra những thiệt hại nghiêm trọng cho các tổ chức hoặc cá nhân.
\subsection{Giải pháp}
\section{Mặt nạ dữ liệu}
\section{Mã hóa dữ liệu}
\section{Trao đổi khóa Diffie-Hellman và Elliptic-curve Diffie-Hellman}
\subsection{Trao đổi khóa Diffie-Hellman}

\ac{DH} là abc abc

\subsection{Mật mã đường cong Elliptic}
\subsection{Trao đổi khóa Diffie-Hellman trên đường cong elliptic}
\section{Hệ mã dòng có xác thực ChaCha20-Poly1305}
\subsection{Hệ mã dòng ChaCha20}
\subsection{Mã xác thực Poly1305}
\section{Hàm băm BLAKE2s}