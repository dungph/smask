
\chapter*{LỜI NÓI ĐẦU}
\phantomsection
\addcontentsline{toc}{chapter}{LỜI NÓI ĐẦU}

\paragraph{}
Ngày nay với sự phát triển mạnh mẽ của công nghệ thông tin, đặt biệt là sự phát triển của mạng Internet, ngày càng có nhiều dữ liệu lớn được sản sinh và lưu trữ, bảo vệ quyền riêng tư và đảm bảo an toàn thông tin ngày càng trở nên cấp thiết hơn bao giờ hết. Trong thực tế nhiều dữ liệ quan trọng và nhạy cảm được lưu trữ trong các hệ thống máy tính, cơ sở dữ liệu, ứng dụng web và ứng dụng di động, tạo ra mối đe dọa đến quyền riêng tư của người dùng.

\paragraph{}
Để giải quyết vấn đề này, mặt nạ dữ liệu (Data Masking) đã trở thành một trong những kỹ thuật phổ biến để bảo vệ dữ liệu nhạy cảm và đảm bảo an toàn thông tin. Kỹ thuật này cho phép ẩn danh hoặc che giấu dữ liệu  trong một tập dữ liệu, giúp đảm bảo những thông tin này không bị tiết lộ và được bảo vệ. 

\paragraph{}
Với mong muốn tìm hiểu về Data Masking, nhóm thực hiện báo cáo với đề tài: “\textbf{Viết chưng trình quản lý dữ liệu an toàn dưới dạng mặt nạ dữ liệu trên SQLite sử dụng ngôn ngữ Rust}”. Với mục tiêu trên, báo cáo bao gồm các chương và bố cục sau:

\hspace{1cm}Chương 1: Tổng quan về an toàn và bảo mật thông tin

\hspace{1cm}Chương 2: Xây dựng chương trình

\hspace{1cm}Chương 3: Kết luận và đánh giá

\vspace{0.5cm}

\hspace{6cm}Hà Nội, ngày 04 tháng 11 năm 2022

\vspace{2cm}

\hspace{7cm}Nhóm sinh viên thực hiện

